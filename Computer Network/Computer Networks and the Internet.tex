\documentclass[UTF8, scheme=plain, 12pt]{ctexrep}
\usepackage[en-US]{datetime2} % English date formatting
\usepackage[svgnames,table]{xcolor} % Add 'table' for more colors
\usepackage{amsmath} % Required for \text command
\usepackage{graphicx}
\usepackage[normalem]{ulem}

\setlength{\parindent}{0pt} % Add this line - removes ALL indentation

\title{\mbox{\textbf{Note for Computer Networks}}}
\author{JohnsonFish Ye}

\begin{document}
\maketitle
\chapter{Computer Networks and the Internet}
\section{What Is the Internet}
\section{The Network Edge}
\section{The Network Core}
  Briefly speaking, this section talks about the mesh of \textbf{packet  switches} (\textbf{router} or \textbf{link-layer-switches}) and \textbf{links interconnecting the Internet's end systems}
  \begin{center}
      \includegraphics[scale=0.5]{The Network Core.png}
  \end{center}

  \subsection{Packet Switching}
  The source end system breaks long messages into smaller chunks of data known as \textbf{packets}

  \vspace{1cm}
  \textcolor{CornflowerBlue}{\textbf{Store-and-Forward Transmission}}\newline

  This means that the packet switch must receive the entire packet before it begin to forward.

  So when a router receive some bit from the source, it has to \textbf{buffer} (store) the bits.\newline

  \textbullet\quad The amount of time that elapses from (ignore propagation delay here)\newline

  Let's consider the general case of sending one packet through N links and each of rate R.\newline

  Thus, there are N - 1 routers. At time L/R, the 1st router begins to forward. So at (N - 1)L/R, the last router begins to forward and the packet will reach the destination at N(L - R).\newline

  \[ d_{end-to-end} = N\frac{L}{R} \] \newline

  What's more, you may think about such two case:\newline

  \textbullet\quad The soure and the switch have different transmission rate

  \textbullet\quad Sending more packets over a series of links.
  \vspace{1cm}

  \textcolor{CornflowerBlue}{\textbf{Queuing Delays and Packet Loss}}\newline

  Each packet switch has multiple links attach to it. And for \textbf{each} attached link, the packet switch has an \textbf{output buffer} (or \textbf{output queue}).\newline

  When a new packet arrives and find the buffer is full, the \textbf{packet loss} occurs - either the arriving packet or one of the already-queued packets will drop.

  \vspace{1cm}

  \textcolor{CornflowerBlue}{\textbf{Forwarding Table and Routing Protocols}}\newline

  In the Internet, every end system has an IP address included in the packet's header.\newline

  \textbullet\quad The address has hierarchical structure like postal address.\newline

  A router use the destination address to index the \textbf{forwarding table} to determine the appropiate outbound link.\newline

  The forwarding table gets set automatically accoring to the routing protocols, which determine the shortest path to each destination.

  We will delve more into the generation of forwarding tables in chapter5 (I should make a hyperlink here then).

  \subsection{Circuits Switching}
  In circuit switching, the resource should be reserved.\newline

  If one link has 4 circuits, then each circuit has 1/4 of the link's total capacity.
  
  \vspace{1cm}

  \textcolor{CornflowerBlue}{\textbf{Multiplexing in Circuit-Switching Networks}}
  \vspace{1cm}

  \textbullet\quad \textbf{Frequency-Division Multiplexing (FDM)}

  \textbullet\quad \textbf{Time-Division Multiplexing (TDM)}\newline

  Time is divided into frames of fixed duration, and each frame is devided into a fixed number of tiem slots.

  Each connection occupies one time slot in every frame.
  \begin{center}
    \includegraphics[scale=0.3]{Presentation of FDM and TDM.png}
  \end{center}

  For TDM, 
  
  \[transmission rate = frame rate \times bit/slot\]\newline

  \textbf{Obviously}, circuit switching is wasteful for some of the dedicated circuits are idle during \textbf{silent periods (quiescent periods)}\newline

  \textcolor{CornflowerBlue}{Let's discuss Packet Switching Versus Circuit Switching}\newline

  \begin{enumerate}
    \item Circuit Switching
    \begin{itemize}
      \item Good at real-time services e.g. telephone calls and video conference acalls
    \end{itemize}
    \item Packet Switching
    \begin{itemize}
      \item Better sharing of transmission capacity.
      \item Simpler, more efficient.
    \end{itemize}
  \end{enumerate}


  Take a look at some numerical examples showing the efficiency of packet switching:\newline

  \begin{itemize}
    \item Total Transmission Rate = 1 Mbps
    \item Each user generates data at a rate of 100 kbps
    \item A user only be active 10\% of the time 
  \end{itemize}

  For TDM, the link can only support (1Mb / 100kb) = 10 simultaneous users.

  If there 35 users, according to probability, there are only 3.5 users being active at a time.

  Thus, just $3.5 \times 100kbs = 350kbps$ is in use while (1Mbps - 350kbps) is idle. 
\newline

  More Simply, if there are 10 users in total but just 1 is active. For TDM, it need 10 times of time to transmit all of its bits compared to the packet switching.\newline

  \uwave{In fact, the trend has certainly been in the direction of packet switching.}



    
\end{document}
