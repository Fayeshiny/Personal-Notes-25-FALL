\documentclass[UTF8, scheme=plain, 12pt]{ctexrep}
\usepackage[en-US]{datetime2} % English date formatting
\usepackage[svgnames,table]{xcolor} % Add 'table' for more colors
\usepackage{amsmath} % Required for \text command
\usepackage{graphicx}
\usepackage[normalem]{ulem}

\setlength{\parindent}{0pt} % Add this line - removes ALL indentation

\title{\mbox{\textbf{Note for Computer Networks}}}
\author{JohnsonFish Ye}

\begin{document}
\maketitle
\chapter{Computer Networks and the Internet}
\section{What Is the Internet}
\section{The Network Edge}
\section{The Network Core}
  Briefly speaking, this section talks about the mesh of \textbf{packet  switches} (\textbf{router} or \textbf{link-layer-switches}) and \textbf{links interconnecting the Internet's end systems}
  \begin{center}
      \includegraphics[scale=0.5]{The Network Core.png}
  \end{center}

  \subsection{Packet Switching}
  The source end system breaks long messages into smaller chunks of data known as \textbf{packets}

  \vspace{1cm}
  \textcolor{CornflowerBlue}{\textbf{Store-and-Forward Transmission}}\newline

  This means that the packet switch must receive the entire packet before it begin to forward.

  So when a router receive some bit from the source, it has to \textbf{buffer} (store) the bits.\newline

  \textbullet\quad The amount of time that elapses from (ignore propagation delay here)\newline

  Let's consider the general case of sending one packet through N links and each of rate R.\newline

  Thus, there are N - 1 routers. At time L/R, the 1st router begins to forward. So at (N - 1)L/R, the last router begins to forward and the packet will reach the destination at N(L - R).\newline

  \[ d_{end-to-end} = N\frac{L}{R} \] \newline

  What's more, you may think about such two case:\newline

  \textbullet\quad The soure and the switch have different transmission rate

  \textbullet\quad Sending more packets over a series of links.
  \vspace{1cm}

  \textcolor{CornflowerBlue}{\textbf{Queuing Delays and Packet Loss}}\newline

  Each packet switch has multiple links attach to it. And for \textbf{each} attached link, the packet switch has an \textbf{output buffer} (or \textbf{output queue}).\newline

  When a new packet arrives and find the buffer is full, the \textbf{packet loss} occurs - either the arriving packet or one of the already-queued packets will drop.

  \vspace{1cm}

  \textcolor{CornflowerBlue}{\textbf{Forwarding Table and Routing Protocols}}\newline

  In the Internet, every end system has an IP address included in the packet's header.\newline

  \textbullet\quad The address has hierarchical structure like postal address.\newline

  A router use the destination address to index the \textbf{forwarding table} to determine the appropiate outbound link.\newline

  The forwarding table gets set automatically accoring to the routing protocols, which determine the shortest path to each destination.

  We will delve more into the generation of forwarding tables in chapter5 (I should make a hyperlink here then).

  \subsection{Circuits Switching}
  In circuit switching, the resource should be reserved.\newline

  If one link has 4 circuits, then each circuit has 1/4 of the link's total capacity.
  
  \vspace{1cm}

  \textcolor{CornflowerBlue}{\textbf{Multiplexing in Circuit-Switching Networks}}
  \vspace{1cm}

  \textbullet\quad \textbf{Frequency-Division Multiplexing (FDM)}

  \textbullet\quad \textbf{Time-Division Multiplexing (TDM)}\newline

  Time is divided into frames of fixed duration, and each frame is devided into a fixed number of tiem slots.

  Each connection occupies one time slot in every frame.
  \begin{center}
    \includegraphics[scale=0.3]{Presentation of FDM and TDM.png}
  \end{center}

  For TDM, 
  
  \[transmission rate = frame rate \times bit/slot\]\newline

  \textbf{Obviously}, circuit switching is wasteful for some of the dedicated circuits are idle during \textbf{silent periods (quiescent periods)}\newline

  \textcolor{CornflowerBlue}{Let's discuss Packet Switching Versus Circuit Switching}\newline

  \begin{enumerate}
    \item Circuit Switching
    \begin{itemize}
      \item Good at real-time services e.g. telephone calls and video conference acalls
    \end{itemize}
    \item Packet Switching
    \begin{itemize}
      \item Better sharing of transmission capacity.
      \item Simpler, more efficient.
    \end{itemize}
  \end{enumerate}


  Take a look at some numerical examples showing the efficiency of packet switching:\newline

  \begin{itemize}
    \item Total Transmission Rate = 1 Mbps
    \item Each user generates data at a rate of 100 kbps
    \item A user only be active 10\% of the time 
  \end{itemize}

  For TDM, the link can only support (1Mb / 100kb) = 10 simultaneous users.

  If there 35 users, according to probability, there are only 3.5 users being active at a time.

  Thus, just $3.5 \times 100kbs = 350kbps$ is in use while (1Mbps - 350kbps) is idle. 
\newline

  More Simply, if there are 10 users in total but just 1 is active. For TDM, it need 10 times of time to transmit all of its bits compared to the packet switching.\newline

  \uwave{In fact, the trend has certainly been in the direction of packet switching.}


  \subsection{A Network of Networks}

  \textbullet\quad Points of Presence (PoPs)\newline

  A PoP is simply a group of one or more routers (at the same location)\newline

  PoPs exist in all levels of the hierarchy, except for the bottom (access ISP).\newline

  \textbullet\quad Any ISP may choose to multi-home, that is, to connect to more than one provider ISPs, so that it can continue to send and receive packets into the Internet even if one of the provider ISPs fails.\newline

  \textcolor{CornflowerBlue}{To save the fee}, ISPs nearby can peer with each other to form a direct connection between them. So that they can communicate with each other without the use of the higher-level providers.\newline

  \textbullet\quad Internet Exchange Point (IXP)\newline

  It is a \textbf{meeting point} where multiple ISPs can peer together.

  \begin{center}
    \includegraphics[scale=0.4]{Structure of A Network of Networks.png}
  \end{center}

\section{Delay, Loss and Throughput in Packet-Switched Networks}

First of all, I prefer to give a summary of the types of delay.
\begin{enumerate}
  \item Processing Delay
  \item Queuing Delay
  \item Transmission Delay
  \item Propogation Delay
\end{enumerate}
  \subsection{Overview of the Delay}

  \textcolor{CornflowerBlue}{\textbf{Processing Delay}}\newline

  The time to examine the packet's \textbf{header} and determine where to \textbf{direct} to packet is one of the component of processing delay.\newline

  Other part: e.g. Check for bit-level errors\newline

  On the order of microseconds or less.\newline

  After this: The router directs the packet to the queue that procedes the link to router B.\newline

  \textcolor{CornflowerBlue}{\textbf{Queuing Delay}}\newline

  The time arriving packets wait in the buffer until it is transmitted onto the link.\newline

  We want to find the function depicting the number of packets in the buffer with repect to the intensity of traffic. (Will be discussed later)\newline

  \textcolor{CornflowerBlue}{\textbf{Transmission Delay}}\newline

  Has already been introduced, no new stuffs here.\newline

  \textcolor{CornflowerBlue}{\textbf{Propagation Delay}}\newline

  The time required to propagate the alread-on-link packet to the target router. (Depend on the physical medium, namely, fiber optics, twisted-pair copper wire etc.)\newline


  Ok, so combining all above, the formula of delay is:\newline

  \[d_{nodal} = d_{proc} + d_{queue} + d_{trans} + d_{prop}\]

\subsection{Queuing Delay and Packet Loss}

The \textbf{Queuing Delay} is the most complicated and interesting component of nodal delay.\newline

Obviously, the queuing delay can vary from packet to packet.\newline
Therefore, we use statistical measures to characterize it.\newline

To talk about the queuing delay, we should first tell if the packets arrive periodically or in bursts.

Suppose R (packet/sec) is the transmission rate, a (packet/sec) is the rate that packets arriveand at the queue and each packet has L bits.\newline

Then the rate at which bits arrive at the queue is La bits/sec.\newline

And we call the ratio \textbf{La/R}, \textcolor{CornflowerBlue}{\textbf{traffic intensity}}\newline

A good designed network should ensure that its \textit{traffic intensity} no greater than 1.\newline

Now let's consider the case La/R $\leq$ 1,\newline

\textbullet\quad If packets arrive periodically - that is, every packet arrives every L/R seconds, then every packet will arrive at an empty queue therefore there will be no queuing delay.\newline

\textbullet\quad If packets arrive in bursts. Suppose N packets arrive simultaneously every (L/R)N seconds.Then the \textcolor{CornflowerBlue}{\textbf{first}} packet will have no queuing delay, but the \textcolor{CornflowerBlue}{\textbf{nth}} packet will have a queuing delay of (n - 1)L/R seconds. \newline

\textbullet\quad However, in practice, packets always arrive randomly but not follow any pattern. The qualitive dependence of average queuing delay on the traffic intensity is shown below:\newline
\begin{center}
  \includegraphics[scale = 0.4]{Dependence of Average Queuing Delay on Traffic Intensity.png}
\end{center}

\textcolor{CornflowerBlue}{\textbf{Packet Loss}}\newline

\textbullet\quad The queue has finite capacity in reality\newline

So if a packet arrive at a queuing finding no space for it, it will be dropped by the router - that is, the packet will be lost.\newline

Form the end-system viewpoint, it looks like a packet having been transmitted into the network core but never emerging from the network at the \textbf{destination}\newline

\subsection{End-to-End Delay}
Amount of the total delay from source to destination\newline

Suppose there are N - 1 routers between source and destination.\newline

The total delay from accumulated nodal delay is given by:\newline

\begin{center}
  \[d_{end-end} = N(d_{proc} + d_{trans} + d_{que} + d_{prop})\]
\end{center}

\subsection{Throughput in Computer Networks}

To define throughput, consider transfering large files from Host A to Host B. \newline

The \textbf{instantaneous throughput} is the rate (bit/s) at which Host B is receiving the file.\newline

If the file consits of F bits and the transfer takes T seconds for Host B to receive them, then the \textbf{average throughput} is F/T bits/sec.\newline

For some applications, delay is not critical, but they are desirable for the highest possible throughput.\newline

\textbf{Now, let's delve more into throughput.}\newline

Let $R_{s}$ denotes the rate of the link between the server and the router and $R_{c}$ denotes the rate of the link between the client and the router. Suppose there is only one bit being sent in the entire network.\newline

\textbf{If $R_{s} < R_{c}$}, then the throughput is $R_{s}$ bps.

\textbf{If $R_{c} < R_{s}$}, then the throughput is $R_{c}$ bps.\newline

So, the throughput is $min\{R_{c}, R_{s}\}$ - the rate of the \textbf{bottleneck link}.\newline

Further, for the N links case, the throughput is $min\{R_{1}, R_{2}, R_{3}, ..., R_{N}\}$.\newline

\begin{center}
  \includegraphics[scale = 0.4]{Throughput for A File Passing Many Routers.png}
\end{center}

For another example, if 10 clients download with 10 servers, let R denotes the rate of the common link.$R_{s} = 2 Mbps, R_{c} = 1 Mbps, R = 5 Mbps$\newline

The common link will divide its rate equally among the 10 downloads, so the bottleneck for each download is now $(5 Mbps / 10 = 500 kbps)$.
\begin{center}
  \includegraphics[scale = 0.4]{10 clients download with 10 servers.png}
\end{center}

So a link with a high rate may also be the bottleneck.\newline

The constraining factor for throughput in today's Internet is typically the access network.

\section{Protocol Layers and Their Service Models}
Each layer provide its service by\newline

(1) Performing certain actions within that layer.\newline
(2) Using the service of the layer directly below it.\newline

\textbf{The ability to change the implementation of one the service without affecting others is one of important advantages of layering.}\newline

\textcolor{CornflowerBlue}{\textbf{Protocol Layering}}\newline

Let's see how each layer provide service the the layer above so-called \textbf{service model} of a layer.\newline

\begin{center}
  \includegraphics[scale = 0.5]{Protocol Stack.png}
\end{center}

\textcolor{CornflowerBlue}{\textbf{Application Layer}}\newline

The packet of information at the application layer is referred to as a \textbf{message}\newline
A special application protocol: \textbf{Domain Name System (DNS)}.\newline

\textcolor{CornflowerBlue}{\textbf{Transport Layer}}\newline
Two transport protocols: TCP and UDP\newline
\textbullet\quad TCP provides a \textbf{connection-oriented} service, which includes guaranteed deliver of application-layer massages to the destination and flow control (that is , sender/receiver speed matching).\newline

\textbullet\quad TCP also breaks long messages into shorter segments and provides a congestion-control mechanism.\newline
So a sour will throttles its transmission rate when there is congestion.\newline


\textbullet\quad UDP protocol provides a \textbf{connectionless} service to its applications.
It's a \textbf{no-frills} service with\newline
\textbullet\quad no reliability\newline
\textbullet\quad no flow control\newline
\textbullet\quad no congestion control\newline

We will refer to a transport-layer packet as a \textbf{segment}.\newline

\textcolor{CornflowerBlue}{\textbf{Network Layer}}\newline

The network layer is responsible for moving network-layer packets known as \textbf{datagrams} from one host to another.

The transport-layer protocol pass a segment and a destination address to the network layer\newline

One of the network-layer protocol is the celebrated \textbf{IP protocol}, which defines the fields in the datagrams as well as how the end systems act on the fields.\newline

\textbf{There is only one IP protocol}.\newline

The network-layer also contains many kinds of \textbf{routing protocols}, which determines the routes that datagrams take between sources and destination.\newline

The network layer is also referred to as \textbf{IP Layer}.\newline

\textcolor{CornflowerBlue}{Link Layer}\newline

Link layer protocols include Ethernet, WiFi and the cable access network's DOCSIS protocol.\newline

Therefore, A datagram may be handled by different link-layer protocols at different links along its route.\newline

The link layer packets are referred to as \textbf{frames}.\newline

\textcolor{CornflowerBlue}{\textbf{Physical Layer}}\newline

The job of physical layer is to move the \textit{individual bit} within the frame from one node to the next.\newline

The protocol of this layer is dependent on the physical medium of the link.\newline

(\textit{For example, Ethernet has many physical-layer protocols: one for twisted-pair copper wire, another for coaxial cable, another for fiber, and so on. In each case, a bit is moved across the link in a different way.})

\subsection{Encapsulation}
\begin{center}
  \includegraphics[scale = 0.3]{Encapsulation along Protocol Layers.png}
\end{center}

Link-layer switch and router do not implement \textit{all} of the layers in the protocol stack.\newline

The process of \textit{Encapulation}:\newline

\begin{enumerate}
  \item The trasport layer take the messages and add \textbf{trapsort-layer header information}.\newline \textbf{Actions}:
  \begin{itemize}
    \item Breaks application data into segments
    \item Adds port numbers
    \item Add information that allows the receiver-side transport layer to deliver the message up to the application layer
    \item Add error-detection bit
    \item Sends them to the network layer.
  \end{itemize}
  \item The Network Layer receives the segments from transport layer and add \textbf{network layer header information}, such as the IP address of source and destination.
  \item The Link Layer receive the datagrams from the Network layer.\newline \textbf{Actions}:
  \begin{itemize}
    \item Wraps packets into frames
    \item Adds MAC addresses
    \item Checks for transmission errors
    \item sends to the physical layer.
  \end{itemize}
  \item The Physical Layer converts bits into electrical, optical, or radio signals for transmission over the medium.
\end{enumerate}

At each layer, a packet has a header field and a \textbf{payload field}.\newline

\section{Networks Under Attack}
\textcolor{CornflowerBlue}{\textbf{Bad Guys Can Put Malware into Hosts Via the Internet.}}\newline
May use a \textbf{botnet} for spam e-mail or distributed \textbf{denial-of-service (DDoS)}.\newline
Most of the malware is \textbf{self-replicating} today.\newline

\textcolor{CornflowerBlue}{\textbf{Bad Guys Can Attack Servers and Network Infrastructure}}\newline

\begin{itemize}
  \item Vulnerability Attack
  \item Bandwidth Flooding (DDoS again)
  \item Connection Flooding
\end{itemize}

\textcolor{CornflowerBlue}{\textbf{Bad Guys Can Sniff Packets}}\newline

A \textbf{Packet Sniffer} - a passive receiver that records a copy of every packet that flies by.

\textcolor{CornflowerBlue}{\textbf{Bad Guys Can Masquerade as Someone You Trust}}\newline

The ability to inject packets into the Internet with a false source address is known as \textbf{IP spoofing}.\newline

What is the consequence?\newline
Imagine the unsuspecting receiver (say an Internet router) who receives such a packet, takes the (false) source address as being truthful, and then performs some command embedded in the packet's contents (say modifies its forwarding table)\newline

In closing this section, We should keep in mind that communication 
among mutually trusted users is the \textbf{exception} rather than the rule. \newline

\textbf{Welcome to the world of modern computer networking!}

    
\end{document}
